%\clearpage
%\chapter*{Abstract}
%\addcontentsline{toc}{chapter}{\protect Abstract}


Această lucrare prezintă proiectarea și %
implementarea unui sistem automatizat pentru %
prepararea ceaiului, utilizând un %
microcontroler bazat pe ATmega328P. Proiectul %
integrează componente electronice accesibile, %
cum ar fi servo motoare, LED-uri, butoane %
și o sursă de alimentare portabilă, %
pentru a demonstra aplicabilitatea %
tehnologiilor moderne în automatizarea %
activităților cotidiene.  

Sistemul controlează un un braț miniatural, %
capabil să ridice %
și să coboare un pliculeț de ceai %
într-o ceașcă, în funcție de un %
temporizator programabil. Controlul motorului %
este realizat prin semnale PWM, asigurând %
o mișcare precisă și repetabilă. %
De asemenea, designul modular permite %
extinderea ulterioară a funcționalităților, %
inclusiv integrarea unor senzori %
suplimentari pentru monitorizarea %
temperaturii și greutății.  

Lucrarea oferă o analiză detaliată %
a configurației componentelor electronice, %
a principiilor de funcționare și a %
etapelor de implementare. Sunt discutate %
problemele întâmpinate pe parcursul %
realizării proiectului, precum și %
soluțiile adoptate pentru  %
performanței sistemului.  

Rezultatele experimentale confirmă %
fezabilitatea soluției propuse și relevă %
oportunități pentru îmbunătățiri viitoare, %
inclusiv adăugarea conectivității wireless %
pentru control de la distanță. Proiectul %
demonstrează utilitatea practică a unui %
sistem simplu și eficient, aplicabil %
atât în scopuri educaționale, cât și %
ca soluție pentru optimizarea %
proceselor casnice.  
