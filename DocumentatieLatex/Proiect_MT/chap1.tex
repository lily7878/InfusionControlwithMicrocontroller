\chapter{Introducere}
\label{chap:intro} 

\section{Justificarea abordării temei}
\label{sec:justification}

Proiectul propus are ca obiectiv dezvoltarea %
unui sistem automatizat pentru prepararea %
ceaiului, utilizând o mini-macara %
controlată printr-un microcontroler ATmega328P %
cu chip CH340G. Alegerea acestei teme este %
justificată de dorința de a demonstra %
aplicabilitatea tehnologiilor moderne de %
automatizare în activități casnice obișnuite. %

Proiectul ilustrează cum un proces %
cotidian poate fi optimizat prin %
integrarea soluțiilor bazate pe \gls{iot} %
și control automatizat. %

Automatizarea procesului de preparare a %
ceaiului oferă un exemplu concret al %
aplicabilității tehnologiilor inteligente, %
care pot simplifica sarcinile zilnice, %
economisind timp și resurse. Această %
abordare contribuie la promovarea %
utilizării microcontrolerelor accesibile %
și ușor de programat, similar cu Arduino, %
pentru dezvoltarea unor soluții %
inovatoare. %

Tema este relevantă și datorită %
interesului crescut pentru dispozitivele %
inteligente și pentru crearea unui mediu %
conectat, unde sarcinile repetitive pot fi %
gestionate eficient cu ajutorul %
automatizării. Astfel, proiectul propune %
o soluție funcțională, ușor de implementat, %
cu aplicabilitate largă în diverse domenii, %
de la uz casnic până la cel educațional. %

Pe lângă beneficiile practice, %
acest proiect a avut un impact semnificativ %
asupra dezvoltării mele personale și %
profesionale. Am dobândit cunoștințe %
extinse despre programarea %
microcontrolerelor, proiectarea %
circuitelor electronice și gestionarea %
proiectelor tehnice. În plus, mi-am %
îmbunătățit abilitățile de rezolvare a %
problemelor și gândirea logică prin %
testarea și optimizarea sistemului. %
Acest proces de învățare practică a %
consolidat înțelegerea conceptelor de %
automatizare și \gls{iot}, oferindu-mi o %
perspectivă mai clară asupra %
potențialului acestor tehnologii.

\section{Importanța și Actualitatea Temei}
\label{sec:importance}

Tema abordată este relevantă în contextul %
actual al tendințelor de automatizare și %
al progresului rapid al tehnologiilor %
integrate. Importanța proiectului rezidă %
în promovarea conceptelor de eficiență %
energetică, confort și control personalizat %
al proceselor domestice. De asemenea, %
creșterea interesului pentru dispozitivele %
inteligente și accesibile face ca acest %
proiect să fie pertinent pentru explorarea %
unor soluții inovatoare la scară redusă. %

Acuitatea temei este subliniată de faptul %
că tehnologiile \gls{iot} devin din ce în ce %
mai prezente în viața cotidiană, iar %
exemplele practice contribuie la adoptarea %
rapidă a acestora în societate. Proiectul %
oferă un exemplu concret și accesibil, %
fiind aplicabil atât în educație, cât și %
în activități casnice. %

\section{Originalitatea și Aplicabilitatea Lucrării}
\label{sec:origsiaplic}

Originalitatea proiectului constă în %
implementarea unui sistem simplu, dar %
eficient, de automatizare, combinând %
componente hardware ușor accesibile cu %
programare software flexibilă. Proiectul %
nu își propune să reinventeze %
automatizarea, ci să aducă o contribuție %
modestă și practică la utilizarea %
tehnologiilor existente. Principiul %
conform căruia progresul tehnologic este %
rezultatul contribuțiilor graduale și %
al îmbunătățirii continue a soluțiilor %
existente constituie baza acestei lucrări. %

Deși nu îmi propun să aduc inovații %
radicale, valoarea proiectului constă în %
adaptarea și integrarea inteligentă a %
tehnologiilor moderne, demonstrând modul %
în care acestea pot fi utilizate pentru %
optimizarea proceselor cotidiene. %

Proiectul demonstrează cum inovațiile %
mici, dar bine gândite, pot îmbunătăți %
procesele cotidiene și pot servi drept %
bază pentru dezvoltări viitoare. Prin %
utilizarea platformei Arduino sau a %
componentelor similare, lucrarea permite %
adaptări și extinderi ulterioare, putând %
fi ușor personalizată în funcție de %
cerințele utilizatorilor. %

Aplicabilitatea lucrării este vastă, %
putând fi folosită atât în scop educativ, %
pentru învățare și experimentare, cât și %
ca soluție concretă pentru optimizarea %
preparării ceaiului în medii rezidențiale %
sau comerciale. Acest proiect poate %
servi, de asemenea, ca bază pentru %
dezvoltarea unor dispozitive similare %
destinate altor procese automatizate din %
gospodărie. %


% A cool list:
% \begin{enumerate}
% 	\item item 1
% 	\item item 2
% 	\item item 3
% \end{enumerate}

% \begin{figure}
% 	\centering
% 	\includegraphics[width=.5\textwidth]{example-image-duck}
% 	\caption{A cool duck}
% 	\label{fig:figure1}
% \end{figure}


