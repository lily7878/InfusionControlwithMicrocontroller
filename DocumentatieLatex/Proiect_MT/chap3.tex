\chapter{Concluzii}

\section{Utilitatea Practică a Lucrării}

Proiectul realizat demonstrează fezabilitatea %
implementării unui sistem automatizat simplu %
și eficient pentru prepararea ceaiului, %
integrând componente electronice accesibile %
și ușor de utilizat. Acesta oferă un %
exemplu concret de aplicare %
a tehnologiilor moderne în optimizarea %
activităților zilnice, evidențiind potențialul %
automatizării în îmbunătățirea confortului %
utilizatorilor.  

Sistemul poate fi folosit atât ca dispozitiv %
practic pentru uz casnic, cât și ca instrument %
educativ pentru învățarea noțiunilor de %
programare, electronice și automatizare. %
Flexibilitatea soluției permite adaptări și %
extinderi viitoare, inclusiv integrarea unor %
senzori suplimentari sau a conectivității %
wireless pentru control la distanță.

\section{Aspecte Economice și Eficiență}

Din punct de vedere economic, sistemul propus %
are un cost de producție redus, datorită %
utilizării unor componente electronice %
comune și ieftine, precum microcontrolerul %
compatibil Arduino, servo motorul continuu %
și \gls{led}-urile. %
 
Acest aspect îl face accesibil atât pentru %
utilizatorii individuali, cât și pentru %
instituțiile educaționale care doresc să %
folosească proiectul în scop didactic.  

Costurile suplimentare pentru îmbunătățiri, %
precum imprimarea \gls{3d} a componentelor %
mecanice sau adăugarea senzorilor de %
greutate, poziție și temperatură, sunt %
moderate și justificate de creșterea %
funcționalității și preciziei sistemului.  

Pe termen lung, dispozitivul poate contribui %
la economisirea timpului utilizatorilor prin %
automatizarea unui proces repetitiv, %
justificând astfel investiția inițială %
prin eficiența obținută.  
